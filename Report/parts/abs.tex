%!TeX root=../main.tex
\clearpage
\pagestyle{empty}
\vspace*{\fill}

\section*{چکیده}

محور اصلی پروژه، طراحی و پیاده‌سازی ایستگاه هواشناسی سینوپتیک (\متن‌لاتین{Synoptic}) متشکل از یک دستگاه که در فاصله چند کیلومتری از مرکز اصلی قرار داده می‌شود تا پارامترهای جوی نظیر دمای هوا، فشار، رطوبت، شدت نور، سرعت و جهت باد را اندازه‌گیری کرده و داده‌ها را از طریق امواج رادیویی به دستگاه دیگر که در پایگاه قرار دارد جهت ثبت و نمایش روی رایانه‌های پایگاه مخابره کند، می‌باشد. در این پروژه از میکروکنترلر \متن‌لاتین{STM32} جهت برنامه‌ریزی و ایجاد ارتباط میان بخش‌های مختلف و همچنین از گیرنده و فرستنده‌های آلتراسونیک جهت محاسبه سرعت و جهت باد استفاده‌شده است. به‌طورکلی با استفاده از سنسورهای دیجیتال در این پروژه علاوه بر کم‌تر شدن حجم دستگاه‌ها مصرف انرژی آن نیز نسبت به ایستگاه‌های سنتی که تجهیزاتی اغلب مکانیکی داشتند کمتر شده است. درنهایت نیز استفاده از ماژول لورا (\متن‌لاتین{LoRa}) به‌عنوان بهینه‌ترین و کم‌هزینه‌ترین روش ارتباط راه گشای انتخاب روش ارسال داده‌های رادیویی خواهدبود. 

\noindent
{\bf
واژه‌های كليدی:
} 
\متن‌لاتین{STM32}، \متن‌لاتین{LoRa}، \متن‌لاتین{Ultrasonic}، \متن‌لاتین{Python}

\vspace*{\fill}
\clearpage