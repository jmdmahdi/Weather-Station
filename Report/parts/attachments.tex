%!TeX root=../main.tex
\clearpage
\phantomsection
\fancyhead[L]{\fontsize{14}{15} \selectfont پیوست}
\chapter*{پیوست}
\addcontentsline{toc}{chapter}{پیوست}

\section*{سورس‌کد ایستگاه}

\subsection*{\متن‌لاتین{main.h}}
\begin{latin}
	\lstinputlisting[language=C, style=codeStyle]{Code/Station/main.h}
\end{latin}

\subsection*{\متن‌لاتین{main.c}}
\begin{latin}
	\lstinputlisting[language=C, style=codeStyle]{Code/Station/main.c}
\end{latin}

\subsection*{\متن‌لاتین{SX1278.h}}\label{SX1278.h}
از این فایل به طور مشترک در سمت سنسور نیز استفاده شده است.
\begin{latin}
	\lstinputlisting[language=C, style=codeStyle]{Code/Station/SX1278.h}
\end{latin}

\subsection*{\متن‌لاتین{SX1278.c}}\label{SX1278.c}
از این فایل به طور مشترک در سمت سنسور نیز استفاده شده است.
\begin{latin}
	\lstinputlisting[language=C, style=codeStyle]{Code/Station/SX1278.c}
\end{latin}

\section*{سورس‌کد سنسور}

\subsection*{\متن‌لاتین{main.h}}
\begin{latin}
	\lstinputlisting[language=C, style=codeStyle]{Code/Sensor/main.h}
\end{latin}

\subsection*{\متن‌لاتین{main.c}}
\begin{latin}
	\lstinputlisting[language=C, style=codeStyle]{Code/Sensor/main.c}
\end{latin}

\subsection*{\متن‌لاتین{SX1278.h}}
این فایل به طور مشترک در بخش ایستگاه نیز استفاده شده است و سورس‌کد آن در بخش \hyperref[SX1278.h]{قبل} آمده است.

\subsection*{\متن‌لاتین{SX1278.c}}
این فایل به طور مشترک در بخش ایستگاه نیز استفاده شده است و سورس‌کد آن در بخش \hyperref[SX1278.c]{قبل} آمده است.

\subsection*{\متن‌لاتین{BMP180.h}}
\begin{latin}
	\lstinputlisting[language=C, style=codeStyle]{Code/Sensor/BMP180.h}
\end{latin}

\subsection*{\متن‌لاتین{BMP180.c}}
\begin{latin}
	\lstinputlisting[language=C, style=codeStyle]{Code/Sensor/BMP180.c}
\end{latin}

\subsection*{\متن‌لاتین{dwt\_delay.h}}
\begin{latin}
	\lstinputlisting[language=C, style=codeStyle]{Code/Sensor/dwt_delay.h}
\end{latin}

\subsection*{\متن‌لاتین{dwt\_delay.c}}
\begin{latin}
	\lstinputlisting[language=C, style=codeStyle]{Code/Sensor/dwt_delay.c}
\end{latin}

\subsection*{\متن‌لاتین{HCSR05.h}}
\begin{latin}
	\lstinputlisting[language=C, style=codeStyle]{Code/Sensor/HCSR05.h}
\end{latin}

\subsection*{\متن‌لاتین{HCSR05.c}}
\begin{latin}
	\lstinputlisting[language=C, style=codeStyle]{Code/Sensor/HCSR05.c}
\end{latin}

\subsection*{\متن‌لاتین{MAX44009.h}}
\begin{latin}
	\lstinputlisting[language=C, style=codeStyle]{Code/Sensor/MAX44009.h}
\end{latin}

\subsection*{\متن‌لاتین{MAX44009.c}}
\begin{latin}
	\lstinputlisting[language=C, style=codeStyle]{Code/Sensor/MAX44009.c}
\end{latin}

\subsection*{\متن‌لاتین{QMC5883L.h}}
\begin{latin}
	\lstinputlisting[language=C, style=codeStyle]{Code/Sensor/QMC5883L.h}
\end{latin}

\subsection*{\متن‌لاتین{QMC5883L.c}}
\begin{latin}
	\lstinputlisting[language=C, style=codeStyle]{Code/Sensor/QMC5883L.c}
\end{latin}

\section*{سورس‌کد برنامه دسکتاپ}

\subsection*{\متن‌لاتین{main.py}}
\begin{latin}
	\lstinputlisting[language=Python, style=codeStyle]{Code/DesktopApp/main.py}
\end{latin}

\subsection*{\متن‌لاتین{db.py}}
\begin{latin}
	\lstinputlisting[language=Python, style=codeStyle]{Code/DesktopApp/db.py}
\end{latin}

\subsection*{\متن‌لاتین{chartWidget.py}}
\begin{latin}
	\lstinputlisting[language=Python, style=codeStyle]{Code/DesktopApp/chartWidget.py}
\end{latin}

\subsection*{\متن‌لاتین{compassWidget.py}}
\begin{latin}
	\lstinputlisting[language=Python, style=codeStyle]{Code/DesktopApp/compassWidget.py}
\end{latin}

\subsection*{\متن‌لاتین{worker.py}}
\begin{latin}
	\lstinputlisting[language=Python, style=codeStyle]{Code/DesktopApp/worker.py}
\end{latin}

\subsection*{\متن‌لاتین{workerSignals.py}}
\begin{latin}
	\lstinputlisting[language=Python, style=codeStyle]{Code/DesktopApp/workerSignals.py}
\end{latin}

\subsection*{\متن‌لاتین{signalWakeupHandler.py}}
\begin{latin}
	\lstinputlisting[language=Python, style=codeStyle]{Code/DesktopApp/signalWakeupHandler.py}
\end{latin}

\subsection*{\متن‌لاتین{mainWindow.ui}}\label{mainWindow.ui}
این فایل به طور خودکار با استفاده از نرم‌افزار \متن‌لاتین{Qt Designer} ساخته شده است.
\begin{latin}
	\lstinputlisting[language=Python, style=codeStyle]{Code/DesktopApp/mainWindow.ui}
\end{latin}

\subsection*{\متن‌لاتین{mainWindow.py}}
این فایل به طور خودکار توسط ابزار \متن‌لاتین{pyuic5} و با استفاده از فایل
\hyperref[mainWindow.ui]{mainWindow.ui}
توسط دستور زیر ساخته شده است: 
\begin{latin}
	pyuic5 -x mainWindow.ui -o ../mainWindow.py
	\lstinputlisting[language=Python, style=codeStyle]{Code/DesktopApp/mainWindow.py}
\end{latin}


