%!TeX root=../main.tex
\فصل{آشنایی با \متن‌لاتین{Git} و \متن‌لاتین{GitHub}}
\قسمت{مقدمه}

برای فهم گیت‌هاب\پانویس{Github}، ابتدا باید با گیت\پانویس{Git} آشنا شویم. گیت یک سیستم کنترل نسخه‌ی متن‌ باز است که توسط خالق لینوکس، لینوس تروالدز، ساخته شد. گیت مانند سایر سیستم‌های کنترل نسخه از قبیل ساب‌ورژن، CVS و مرکوریال است؛ اما اساسا سیستم کنترل نسخه چیست؟ وقتی توسعه‌دهندگان چیز جدیدی مانند یک اپلیکیشن خلق می‌کنند، مدام تغییراتی در کدهای آن می‌دهند تا اولین نسخه‌ی رسمی و غیر بتا آماده‌ی انتشار شود. این روند در به‌روزرسانی برای نسخه‌های بعدی نیز ادامه دارد.

\begin{figure}[!h]
	\centering
	\includegraphics[width=0.7\linewidth]{Assets/gitandgithub.jpg}
\end{figure}


سیستم‌های کنترل نسخه تغییرات و بازنگری‌های توسعه‌دهندگان را در یک مخزن مرکزی ذخیره می‌کنند. با این کار همکاری بین توسعه‌دهندگان آسان می‌شود؛ به این شکل که هر توسعه‌دهنده می‌تواند نسخه‌ی جدید را دانلود کند، تغییرات را در آن اعمال و سپس آپلود کند. کلیه‌ی توسعه‌دهندگان قادر به مشاهده‌ی تغییرات جدید، دانلود آن‌ها و مشارکت در پروژه خواهند بود.

در دنیای رایانه، هاب به تجهیزات سخت‌افزاری گفته می‌شود که از آن اشتراک‌گذاری شبکه با گجت‌های مختلف استفاده می‌شود و در نتیجه هاب به نوعی شبکه را گسترش می‌دهد. هاب در گیت‌هاب نیز چنین مفهومی دارد. توسعه‌دهندگان پروژه‌های خود را در گیت‌هاب ذخیره می‌کنند و از این طریق به شبکه‌ی عظیم توسعه‌دهندگان دنیا وصل می‌شوند. در گیت‌هاب این امکان وجود دارد که پروژه‌ای را از مخزن توسعه‌دهنده به مخزن خود کپی و در آن تغییرات اعمال کنید و سپس درخواست اعمال تغییرات را به صاحب پروژه بفرستید تا در پروژه‌ی اصلی اعمال کند. امکان پرسش و پاسخ نیز در این شبکه‌ی گیت فراهم است.

حالا که با مفاهیم گیت و گیت‌هاب آشنا شدیم به نحوه‌ی کار با این سرویس‌ها می‌پردازیم.

\قسمت{آموزش کار با گیت}
\زیرقسمت{نیازهای بنیادی}
قبل از هرچیز لازم است مواردی را نصب کنید. برای این‌کار نسخه‌ی متناسب با سیستم‌عامل خود را از اینجا دانلود و نصب کنید. اگر از لینوکس استفاده می‌کنید، از طریق پکیج منیجر نیز می‌توانید اقدام کنید.

در مرحله‌ی بعد، از آن‌جایی که در روند آموزش، یک مخزن شامل یک کد و یک \متن‌لاتین{README} خواهیم ساخت، یک دایرکتوری برای آن در نظر بگیرید.

پس از آن، به عملیات معمولی نظیر \متن‌لاتین{init}، کلون، کامیت و \متن‌لاتین{diff} می‌پردازیم. البته، عملیات پیشرفته‌تری نیز وجود دارد که در مراحل اولیه نیازی به آن‌ها نخواهید داشت.

\زیرقسمت{راه‌اندازی یک مخزن (\متن‌لاتین{Repository})}
قبل از شروع کار با گیت، باید یک مخزن پروژه راه‌اندازی کنید تا به کمک گیت آن را مدیریت کنید. ترمینال را باز کنید و در دایرکتوری پروژه‌ی خود دستور \متن‌لاتین{git init} را وارد کنید.

با این کار یک دایرکتوری مخفی با نام \متن‌لاتین{git.} در دایرکتوری پروژه‌ی شما ساخته خواهد شد. این دایرکتوری همان مسیری است که گیت دیتابیس و تنظیمات خود را در آن ذخیره می‌کند تا تغییرات پروژه‌ی شما را دنبال کند.

\زیرقسمت{کلون یا کپی کردن یک مخزن}
راه دیگری برای دسترسی به مخزن وجود دارد که به کلونینگ مشهور است. درست مثل بررسی مخزن در سایر سیستم‌ها، اجرای کد \متن‌لاتین{git clone <repository URL> } یک کپی کامل از مخزن مورد نظر به سیستم شما منتقل خواهد کرد. سپس، می‌توانید تغییرات دلخواه را در آن اعمال کنید. روند اعمال تغییرات ساخت تغییرات، اعمال موقت آن‌ها بدون وارد کردن در مخزن اصلی (\متن‌لاتین{staging})، اعمال تغییر در مخزن یا کامیت (\متن‌لاتین{commit}) را شامل می‌شود.

\متن‌لاتین{افزودن فایل جدید}
در این مرحله می‌توان از زبان‌های برنامه‌نویسی مختلف مانند پایتون، روبی، \متن‌لاتین{Go} یا هر زبان دیگری استفاده کرد. ما در این آموزش از زبان \متن‌لاتین{Python} که معمول‌تر است استفاده می‌کنیم. فایلی به نام \متن‌لاتین{main.py} را در دایرکتوری خود ایجاد و کد زیر را در آن وارد کنید.

\begin{latin}
	print("Hello World")
\end{latin}

بعد از ذخیره‌ی فایل، دستور \متن‌لاتین{\متن‌لاتین{git status}} را از ترمینال اجرا کنید. این دستور وضعیت حاضر مخزن کار شما را نشان می‌دهد. نتیجه‌ی به نمایش درآمده باید مشابه تصویر زیر باشد که در آن \متن‌لاتین{main.py} به‌عنوان یک فایل \متن‌لاتین{untracked} یا بررسی‌نشده فهرست شده است.

حالا طرز کار با چند فایل بدون اعمال تغییرات در مخزن را بررسی می‌کنیم. برای این‌کار یک فایل دوم به نام \متن‌لاتین{\متن‌لاتین{REDME.md}} بسازید. در این فایل جزئیاتی مثل نام پروژه، نام و نشانی ایمیل خود را وارد کنید. دستور \متن لاتین{\متن‌لاتین{git status}} را مجددا اجرا کنید. خواهید دید که این‌بار دو فایل به‌عنوان بررسی‌نشده فهرست شده‌اند.

حالا می‌خواهیم \متن‌لاتین{main.py} را به‌اصطلاح استیج (\متن‌لاتین{stage}) کنیم. فایلی که استیج می‌شود؛ یعنی تغییرات آن انجام گرفته اما هنوز در مخزن اصلی اعمال نشده است. برای این‌کار دستور git add \متن‌لاتین{main.py} را وارد کنید. حالا، دستور وضعیت گیت (\متن‌لاتین{git status}) را مجددا اجرا کنید. خواهید دید که \متن‌لاتین{main.py} به‌عنوان فایلی جدید در بخش تغییرات در انتظار اعمال (\متن‌لاتین{changes to be commited}) فهرست شده است و \متن‌لاتین{\متن‌لاتین{REDME.md}} در همان بخش \متن‌لاتین{Untracked files} قرار دارد.

\زیرقسمت{تنظیمات}
در این مرحله همه‌چیز آماده‌ی اعمال تغییرات یا \متن‌لاتین{commit} است؛ اما قبل از این‌کار باید با تنظیمات ویرایشگر که گیت هنگام نوشتن پیام‌های کامیت مورد استفاده قرار می‌دهد آشنا شوید.

اگر از لینوکس استفاده می‌کنید گیت به‌طور پیش‌فرض، از برنامه‌‌هایی مانند \متن‌لاتین{pico}، \متن‌لاتین{vi}، \متن‌لاتین{vim} یا \متن‌لاتین{emacs} استفاده خواهد کرد. اگر با این برنامه‌ها آشنایی ندارید، ممکن است بخواهید آن‌ها را با نرم‌افزاری مثل \متن‌لاتین{Notepad}، \متن‌لاتین{TextEdit} یا \متن‌لاتین{Gedit} عوض کنید. برای این‌کار دستور زیر را از ترمینال اجرا کنید:

\begin{latin}
	git config --global core.editor <your app's name
\end{latin}
در قسمت آخر کد به جای \متن‌لاتین{your app's name} نام نرم‌افزار مورد نظر خود را وارد کنید.

تنظیمات دیگری مانند تغییر نام و ایمیل و چگونگی نمایش پیام کامیت نیز قابل انجام است. ما در این آموزش از \متن‌لاتین{vim} به‌عنوان ادیتور استفاده می‌کنیم؛ اما شما می‌توانید انتخاب متفاوت خود را داشته باشید.

\زیرقسمت{اعمال اولین تغییر}
کامیت در گیت شباهت بسیار زیادی با کامیت در سایر سیستم‌های کنترل نسخه مانند ساب‌ورژن دارد. روند کار به این شکل است که کار را آغاز می‌کنید و پیامی جهت توضیح اینکه دلیل تغییر انجام گرفته چیست وارد می‌کنید و فایل تغییر می‌یابد. پس دستور \متن‌لاتین{git commit} را اجرا کنید. با این کار ویرایشگر به‌صورت خودکار باز می‌شود و الگوی زیر را نمایش می‌دهد.

\begin{latin}
	\singlespacing
	\noindent
\# Please enter the commit message for your changes. Lines starting\\
\# with '\#' will be ignored, and an empty message aborts the commit.\\
\# On branch master\\
\#\\
\# Initial commit\\
\#\\
\# Changes to be committed:\\
\#       new file: ‌main.py\\
\#\\
\# Untracked files:\\
\#       \متن‌لاتین{REDME.md}\\
\#\\
\end{latin}

با بررسی مداوم وضعیت تغییرات اعمال‌شده توسط دستور \متن‌لاتین{git status} از شرایط مخزن خود آگاهی پیدا خواهید کرد. با این‌کار همواره خواهید دانست چه تغییری را اعمال کرده و چه تغییری را هنوز اعمال نکرده‌اید. یک پیام کامیت خوب باید شامل دو بخش باشد؛ اول این‌که کوتاه و در حد ۷۲ کاراکتر باشد و به‌طور خلاصه تغییر اعمال‌شده را اعلام کند. دیگر این‌که دارای توضیحی بلندتر باشد که به‌طور مجزا در سطری دیگر جزئیات تغییر اعمال‌شده را توضیح دهد. البته مورد دوم اختیاری است و الزامی برای نوشتن آن وجود ندارد.

ما در این مرحله نیاز به نوشتن توضیح پیچیده‌ای نداریم؛ چرا که تنها یک فایل را به مخزن اضافه کرده‌ایم؛ اما چنان‌چه تغییری که اعمال می‌کنید دارای الگوریتم‌های پیچیده‌ای باشد، لازم است توضیحاتی در این بخش برای مطالعه‌ی سایر توسعه‌دهندگان بنویسید و آن‌ها را از چرایی اعمال این تغییر آگاه سازید. بنابراین، پیام ساده‌ی زیر را در ویرایشگر وارد و ذخیره کنید و خارج شوید.

\begin{latin}
	\singlespacing
	\noindent
	“Adding the core script file to the repository”
\end{latin}

\noindent
حالا که تغییرات اعمال شدند. وضعیت گیت را مجددا بررسی کنید. خواهید دید که \متن‌لاتین{\متن‌لاتین{\متن‌لاتین{REDME.md}}} همچنان در قسمت \متن‌لاتین{untracked} قرار دارد.

\زیرقسمت{مشاهده‌ی تغییرات}
در این مرحله از آن‌جایی که چند فایل در قسمت کنترل نسخه داریم و با دستورهای پایه آشنا شده‌ایم، به بررسی تغییرات می‌پردازیم. برای بررسی تغییرات اعمال‌شده در یک فایل از دستور \متن‌لاتین{git diff} استفاده می‌کنیم. این دستور مشابه دستور Linux diff دو فایل را با هم مقایسه می‌کند و تغییرات فایل جدیدتر را نمایش می‌دهد.

در اینجا برای مشاهده‌ی تغییرات فایل \متن‌لاتین{REDME.md} دستور \متن‌لاتین{git diff \متن‌لاتین{REDME.md}} را اجرا می‌کنیم. با این‌کار تغییرات جدیدترین نسخه نسبت به اولین نسخه به نمایش درمی‌آید.

به خاطر داشته باشید که به‌طور پیش‌فرض، دستور \متن‌لاتین{git diff} تغییرات را نسبت به فایل اولیه نشان می‌دهد، نه فایل استیج‌شده. اگر می‌خواهید تغییرات استیج‌شده را مشاهده کنید، دستور \متن‌لاتین{git diff --cached \متن‌لاتین{REDME.md}} را اجرا کنید. این دستور چیزی شبیه کد زیر را به نمایش درخواهد آورد.
\begin{latin}
	\singlespacing
	\noindent
	diff --git a/\متن‌لاتین{REDME.md} b/\متن‌لاتین{REDME.md} new file mode 100644 index 0000000..27c0a86\\ --- /dev/null +++ b/\متن‌لاتین{REDME.md} @@ -0,0 +1,5 @@ +\# Simple Git Project + +\#\#\\ Authors + +Matthew Setter <matthew@maltblue.com>
\end{latin}

در کد نمایش داده‌شده به پنج خط آخر دقت کنید. قبل از هر سطر یک علامت + وجود دارد. این علامت نشانگر افزودن چیزی به فایل است. در اینجا ما فقط اضافه کرده‌ایم؛ اما اگر چیزی حذف کرده بودیم علامت منفی (-) نمایش داده می‌شد.

\زیرقسمت{نکات مهم درباره‌ی استیجینگ یا ایندکس}
اگر مبتدی هستید، یکی از گیج‌کننده‌ترین قسمت‌ها برای شما محیط استیجینگ و رابطه‌ی آن با کامیت خواهد بود.

کامیت در واقع ثبت سوابق تغییرات فایل‌ها نسبت به آخرین تغییر اعمال‌شده است. یعنی شما تغییری در مخزن خود اعمال می‌کنید و به کیت می‌گویید آن فایل‌ها را در یک کامیت ثبت سابقه کند.

کامیت‌ها ماهیت پروژه‌ی شما را در مراحل مختلف حفظ می‌کنند تا هر زمان که نیاز شد بتوانید به مرحله‌ی قبل برگردید.

اما چطور به گیت می‌گویید کدام فایل‌ها را در کامیت قرار دهد؟ اینجا است که استیجینگ یا ایندکس ایفای نقش می‌کند. برای اضافه کردن یک فایل در کامیت، ابتدا باید آن را به محیط استیجینگ اضافه کنید. برای انجام این کار می‌توانید از دستور \متن‌لاتین{>git add <filename} استفاده کنید. پس از اینکه که فایل‌های مورد نظر خود را با این دستور به محیط استیجینگ اضافه کردید، می‌توانید با دستور \متن‌لاتین{git commit} آن‌ها را به‌عنوان کامیت در مخزن اعمال کنید.

\متن‌لاتین{برنچینگ یا شاخه‌بندی}
تا این‌جا با نحوه‌ی شروع کار و اعمال تغییرات و بررسی آن‌ها آشنا شدیم. حالا با مفهوم پیشرفته‌تری به نام شاخه‌بندی آشنا می‌شویم. وقتی به‌طور تیمی روی یک نرم‌افزار کار می‌کنیم. آزمون و خطاهای هر برنامه‌نویس روی شاخه‌ی اصلی کدهای یک برنامه ممکن است دردسرساز شود. گیت این اجازه را به شما می‌دهد که شاخه‌ی اختصاصی خود را داشته باشید. در این حالت، وقتی روی ساخت یک ویژگی کار می‌کنید، آزمایش‌های شما صدمه‌ای به شاخه‌ی اصلی نمی‌زند و می‌توانید وقتی به نتیجه رسیدید، مجموعه‌ی تغییرات اعمال‌شده را با شاخه‌ی اصلی تلفیق یا \متن‌لاتین{merge} کنید.

تا این‌ بخش از آموزش در حال کار روی شاخه‌ی اصلی یا مستر برنچ بوده‌ایم. شاخه‌ی اصلی در واقع همان شاخه‌ای است که به‌صورت پیش‌فرض راه‌اندازی گیت با آن آغاز می‌شود. اکنون قصد داریم یک شاخه به نام \متن‌لاتین{development} (به معنی توسعه) راه‌اندازی کنیم. از ترمینال دستور \متن‌لاتین{git checkout -b develop} را اجرا کنید تا شاخه‌ای به نام \متن لاتین{develop} ساخته شود. اجرای این دستور علاوه بر ساخت شاخه‌ی مذکور بررسی آن را نیز اعمال می‌کند. این شاخه در ابتدا یک کپی از شاخه‌ی اصلی است. یعنی اگر دستور \متن‌لاتین{git status} را اجرا کنید همان دو تغییر اعمال‌شده در \متن‌لاتین{REDME.md} را مشاهده خواهید کرد. حالا فرض کنید می‌خواهیم همین دو کامیت را در شاخه‌ی اصلی تلفیق کنیم.

برای انجام این‌کار ابتدا باید مشخص کنید که قصد تلفیق تغییرات موجود در کدام شاخه را دارید.

پس، دستور \متن‌لاتین{git checkout master} را اجرا کنید. حالا باید تغییرات را از شاخه‌ای که در حال کار روی آن بوده‌اید در این شاخه تلفیق کنید. برای این‌کار دستور \متن‌لاتین{git merge develop} را اجرا کنید.

وقتی کار تمام شد، پیامی مبنی بر تغییر فایل‌ها و خلاصه گزارشی از آن تغییرات نشان داده خواهد شد.

\قسمت{آموزش کار با گیت‌هاب}

پس از آشنایی مقدماتی با گیت وقت آن رسیده است که به گیت‌هاب بپردازیم. همان‌طور که گفته شد، گیت‌هاب چیزی فراتر از یک مخزن پروژه است. این سرویس یکی از نسخه‌های گرافیکی گیت به شمار می‌رود. در واقع، اگر کار با گیت را بلد باشید به‌ندرت در کار با گیت‌هاب به مشکل برخواهید خورد. مسئله‌ی دیگر این است که ممکن است نخواهید پروژه‌ی خود را با دیگران به اشتراک بگذارید. در این حالت استفاده از گیت و ذخیره‌ی تغییرات به‌صورت لوکال بهترین گزینه است؛ همچنین در استفاده از گیت‌هاب شما ملزم به داشتن اتصال اینترنت هستید؛ ولی در کار با گیت چنین الزامی وجود ندارد. با تمام این اوصاف، ممکن است بخواهید صفر تا صد پروژه‌ی خود را در گیت‌هاب انجام دهید؛ چرا که کار با گیت مرارت‌های خاص خود را دارد. برای مثال باید کل منبع را دانلود کنید، سپس وایرایش‌های خود را در قالب یک پچ تهیه کنید و به طریقی مثل ایمیل به سازنده‌ی اصلی بدهید و او این پچ را که از سازنده‌ی آن اطلاع دقیقی ندارد بررسی و در صورت امکان استفاده کند. خواهید دید که در گیت‌هاب به‌عنوان یک شبکه‌ی برخط و متن‌ باز هیچ‌یک از این مرارت‌ها وجود نخواهد داشت؛ اما قبل از شروع کار با گیت‌هاب نیاز به آشنایی با مفاهیم خاصی است که در زیر به آن‌ها می‌پردازیم.

\زیرقسمت{ساخت مخزن یا \متن‌لاتین{Repository}}
مخزن یا \متن‌لاتین{Repository} که به اختصار به آن \متن‌لاتین{repo} نیز گفته می‌شود می‌تواند شامل فولدر، فایل، تصویر، ویدیو و هر آنچه پروژه‌ی شما به آن نیاز دارد باشد. گیت‌هاب در ابتدای ساخت پروژه امکان افزودن \متن‌لاتین{README} و سایر موارد مانند لایسنس را در اختیار می‌گذارد.

مخزن اول شما با نام \متن‌لاتین{hello-world} می‌تواند مکانی برای ذخیره کردن ایده‌ها، منابع یا حتی اشتراک‌گذاری و بحث در مورد چیزهای مختلف باشد.

برای ساخت یک مخزن جدید در گوشه‌ی بالا سمت راست و نزدیک به آواتار یا نماد کاربری شما، روی + و سپس \متن‌لاتین{New repository} کلیک کنید.
توضیح کوتاهی بنویسید.
در صورت تمایل به اضافه کردن \متن‌لاتین{README} گزینه‌ی \متن‌لاتین{Initialize this repository with a README} را انتخاب کنید.
رو \متن‌لاتین{Creat repository } کلیک کنید.

\زیرقسمت{ساخت شاخه یا \متن‌لاتین{Branch}}
برنچینگ روشی برای کار همزمان روی نسخه‌های مختلف یک مخزن است.
به‌طور پیش‌فرض مخزن شما یک شاخه به نام master دارد که شاخه‌ی اصلی به‌حساب می‌آید. از شاخه‌ها برای آزمون و خطا و ویرایش کدها قبل از اعمال تغییرات در شاخه‌ی اصلی استفاده می‌شود.
وقتی یک شاخه بر پایه‌ی شاخه‌ی اصلی می‌سازید، یک کپی از آن با آخرین تغییراتی که تا آن لحظه اعمال‌شده در شاخه‌ی جدید ایجاد می‌شود. اگر زمانی که شما روی شاخه‌ی خود کار می‌کنید کسی کامیتی به شاخه‌ی اصلی اضافه کند می‌توانید آن را در شاخه‌ی خود اعمال کنید.
برای ساخت یک شاخه‌ی جدید به مخزن جدیدی که با نام \متن‌لاتین{hello-world} ساخته‌اید بروید.
روی فهرست بازشونده‌ی موجود در بالای فهرست فایل‌ها که روی آن نام شاخه نوشته شده است کلیک کنید. در تکست‌باکس بازشده نام شاخه‌ی جدید، مثلا \متن‌لاتین{readme-edits} را وارد کنید.
روی دکمه‌ی \متن‌لاتین{Create branch} کلیک کنید یا دکمه‌ی \متن‌لاتین{Enter} را در کیبورد خود بزنید. حالا دو شاخه دارید؛ یکی \متن‌لاتین{master} و دیگری \متن‌لاتین{readme-edits} که کاملا شبیه به هم هستند؛ البته تا زمانی که تغییری در هیچ‌یک اعمال نکرده‌ایم.

\زیرقسمت{ایجاد تغییرات و اعمال آن‌ها}
حالا که شاخه‌ی جدیدی با محتویات یکسان با شاخه‌ی اصلی داریم، بدون ترس از خرابکاری ویرایش‌های خود را آغاز می‌کنیم.
در گیت‌هاب به تغییرات اعمال‌شده کامیت (\متن‌لاتین{commit}) می‌گویند. هر کامیت یک پیام کامیت نیز به همراه دارد که توضیح کوتاهی در رابطه با دلیل اعمال آن تغییر است. پیام‌های کامیت به سایر مشارکت‌کنندگان در پروژه این امکان را می‌دهد که متوجه شوند شما چه تغییری را به چه دلیل اعمال کرده‌اید.
برای اعمال یک تغییر روی فایل \متن‌لاتین{REDME.md} کلیک کنید.
روی آیکون مداد در گوشه‌ی بالا سمت راست کلیک کنید تا بتوانید آن را ویرایش کنید.
در ویرایشگر، کمی درباره‌ی خود بنویسید.
یک پیام کامیت برای توصیف تغییرات خود بنویسید.
روی دکمه‌ی \متن‌لاتین{Commit changes} کلیک کنید.


تغییرات ایجادشده در فایل \متن‌لاتین{README} تنها در شاخه‌ی \متن‌لاتین{readme-edits} اعمال‌ شده‌اند. حالا این شاخه دارای تغییراتی نسبت به شاخه‌ی \متن‌لاتین{master} است.

\زیرقسمت{ایجاد درخواست اعمال تغییرات یا \متن‌لاتین{pull request}}
حالا برای اعمال این تغییرات در شاخه‌ی اصلی باید درخواست آن را ایجاد کنید. \متن‌لاتین{pull request} هسته‌ی تمام همکاری‌ها در گیت‌هاب را تشکیل می‌دهد. با ارسال چنین درخواستی، شما از صاحب شاخه می‌خواهید که تغییرات انجام‌شده توسط شما را بررسی و در صورت صلاحدید به شاخه‌ی اصلی اضافه کند. درخواست‌های اعمال تغییرات تغییرات بین دو شاخه را نمایش می‌دهند. تغییرات، حذف و اضافه‌ها در رنگ‌های سبز و قرمز نشان داده می‌شوند.
با استفاده از سیستم منشن در گیت‌هاب می‌توانید در سیستم پیام‌ \متن‌لاتین{pull request} خود از سایر افراد یا تیم‌ها درخواست کنید که درباره‌ی ویرایش‌های شما نظر بدهند.
شما می‌توانید درخواست اعمال تغییرات را برای خود نیز ارسال کنید. یعنی وقتی تغییری را به‌طور کامل در شاخه‌ی فرعی انجام دادید، می‌توانید برای تلفیق آن در شاخه‌ی اصلی از \متن‌لاتین{pull request} استفاده کنید.

\زیرقسمت{پذیرش درخواست تلفیق تغییرات اعمال‌شده}
در این گام پایانی، نحوه‌ی تلفیق تغییرات شاخه‌ی فرعی در شاخه‌ی \متن‌لاتین{master} را بررسی می‌کنیم.
روی دکمه‌ی سبزرنگ \متن‌لاتین{Merge pull request} کلیک کنید تا تغییرات شما در شاخه‌ی اصلی اعمال شوند.آموزش گیت‌هاب
روی \متن‌لاتین{Confirm merge} کلیک کنید.
حالا که تغییرات را اعمال کرده‌اید، می‌توانید با استفاده از دکمه‌ی \متن‌لاتین{Delete branch} شاخه‌ی فرعی را حذف کنید.


\متن‌لاتین{تعدادی از اصطلاحات رایج در گیت‌هاب}
کد (Code): حالت نمایشی که به‌صورت پیش‌فرض در آن قرار دارید و فایل‌های پروژه به شما نمایش داده می‌شوند.

\noindent
مسائل (Issues): چنان‌چه شما یا هم‌تیمی‌های شما بخواهند مشکلی را در نرم‌افزار گزارش کنند، یا درخواست افزودن قابلیت یا مسائلی این‌چنینی را مطرح کنند، از این گزینه استفاده می‌کنند.

\noindent
ویکی (Wiki): امکانی است برای ثبت جزئی‌تر پروژه نسبت به آن‌چه در \متن‌لاتین{REDME.md} می‌آید.

\noindent
ضربان (Pulse): خلاصه‌ای از آمار پروژه شامل مسائل مطرح‌شده، حل‌شده و حل‌نشده که نشانگر میزان فعال بودن پروژه است.

\noindent
نمودارها (Graphs): پیشرفت پروژه در طول زمان شامل روزهای پرکار و زمان‌هایی که پروژه رها شده و بی‌تغییر مانده است نشان می‌دهد.

\noindent

\قسمت{آموزش انتقال پروژه از گیت به گیت‌هاب}
حالا قصد داریم پروژه‌ی کوچکی را که در گیت روی آن کار کرده بودیم  در گیت‌هاب بارگذاری کنیم. برای این‌کار ابتدا نیاز به ساخت یک حساب کاربری در گیت‌هاب دارید. توجه داشته باشید در روند ساخت حساب کاربری در گیت‌هاب پس از وارد کردن نام کاربری، ایمیل و پسورد، دو تعرفه پیش روی شما قرار می‌گیرد. در گزینه‌ی اول استفاده از گیت‌هاب رایگان خواهد بود اما نمی‌توانید پروژه‌ی محرمانه بسازید. طبعا گزینه‌ی دوم پولی و با امکان ساخت پروژه‌ی محرمانه یا خصوصی است.

پس از ورود به حساب کاربری خود برای بارگذاری پروژه روی علامت مثبت موجود در بالا گوشه‌ی راست کلیک کنید و در فهرست بازشده \متن‌لاتین{New repository} را برای راه‌اندازی مخزن جدید انتخاب کنید. در این مرحله فرم مربوط به ساخت پروژه‌ی جدید ظاهر خواهد شد.

در قسمت \متن‌لاتین{Repository name} یک نام برای مخزن خود وارد کنید. این نام می‌تواند \متن‌لاتین{first-project} به معنی اولین پروژه باشد. می‌توانید توضیحی نیز در خصوص آن ذکر کنید. مثلا \متن‌لاتین{my first Github project} به معنی اولین پروژه‌ی من در گیت‌هاب. حالت پروژه با توجه به نوع کاربری شما، public خواهد بود. نهایتا تیک \متن‌لاتین{inilialize this repository with a README} را بزنید و دو قسمت دیگر را به همان شکل روی \متن‌لاتین{} Noneرها کنید. حالا روی \متن‌لاتین{Create repository} کلیک کنید.

در این مرحله به صفحه‌ی راه‌انداری سریع هدایت خواهید شد. این صفحه امکان بارگذاری پروژه‌ی گیت را به ما خواهد داد. برای انجام این‌کار اولین سطر زیر \متن‌لاتین{or push an existing repository from the command line...} را کپی کنید و آن را در ترمینال یعنی همان‌جایی که تا کنون روی آن کار می‌کردیم پیست کنید. این‌کار به ما خروجی نخواهد داد. حالا همین کار را برای سطر دوم انجام دهید. در این مرحله تغییرات ما به گیت‌هاب منتقل می‌شود و خروجی مشابه آن‌چه در تصویر زیر می‌بینید خواهد بود. حالا در مرورگر خود صفحه‌ی گیت‌هاب را ریفرش کنید. فایل‌های \متن‌لاتین{REDME.md} و \متن‌لاتین{main.py} به‌عنوان اولین فایل‌های پروژه نمایش داده می‌شوند و محتویات \متن‌لاتین{REDME.md} در پایین صفحه نمایان می‌شود.

