%!TeX root=../main.tex
\فصل{مقدمه}
\begingroup

\titleformat{\section}[display]{\normalfont\huge\bfseries}{}{0pt}{}

\قسمت{مقدمه}

ایستگاه هواشناسی، مرکزی مجهز به تجهیزات و ابزارهایی برای اندازه‌گیری‌های جوی است که به ارائه اطلاعات برای پیش‌بینی و مطالعه آب‌وهوا می‌پردازد. اندازه‌گیری انجام‌شده معمولاً شامل دما، فشار هوا، رطوبت، سرعت باد، جهت باد و مقدار بارش است. مشاهدات دستی حداقل یک بار در روز انجام می‌شود، درحالی‌که اندازه‌گیری خودکار حداقل یک بار در ساعت انجام می‌پذیرد.

ایستگاه‌های هواشناسی معمولی مجهز به ابزارهای زیر هستند \مرجع{wikipedia:Weather_station}:

\شروع{فقرات}
\فقره
رطوبت‌سنج برای اندازه‌گیری رطوبت
\فقره
فشارسنج برای اندازه‌گیری فشار جو
\فقره 
دماسنج برای اندازه‌گیری دمای هوا
\فقره
پیرانومتر برای اندازه‌گیری تشعشعات خورشیدی
\فقره
باران‌سنج برای اندازه‌گیری میزان بارش باران در طی یک دوره زمانی مشخص
\فقره
تجهیزاتی نظیر بادسنج، پرچم باد یا جوراب باد برای اندازه‌گیری سرعت و جهت باد
\پایان{فقرات}

ایستگاه‌های پیشرفته‌تر همچنین ممکن است شاخص فرابنفش، رطوبت برگ، رطوبت خاک، دمای خاک، دمای آب در حوضچه‌ها، دریاچه‌ها، نهرها یا رودخانه‌ها و گاهی داده‌های دیگر را اندازه‌گیری کنند. به‌جز دستگاه‌هایی که نیازمند تماس مستقیم با عناصر مورداندازه‌گیری هستند (نظیر بادسنج)، دیگر سنسورها و دستگاه‌ها باید در محفظه‌ای به‌دوراز تابش مستقیم خورشید و وزش باد قرار بگیرند.

ایستگاه‌های هواشناسی سینوپتیک (\متن‌لاتین{Synoptic}) 24 ساعته به‌صورت خودکار هر سه ساعت به سه ساعت پارامترهای جوی را پس از اندازه‌گیری و جمع‌آوری از طریق شبکه‌های مخابراتی منتقل می‌کنند. به‌طور مشابه ایستگاه‌هایی با نام متار (\متن‌لاتین{Metar}) این کار را هر  یک ساعت انجام می‌دهند. وظیفه این ایستگاه‌ها جمع‌آوری اطلاعات جوی از محدوده‌هایی وسیع و مخابره به ایستگاه‌های اصلی به‌منظور اطلاع از وضعیت حال و گذشته و پیش‌بینی شرایط آب و هوایی مناطق در آینده است. 

هدف این پروژه پیاده‌سازی نوعی ایستگاه هواشناسی سینوپتیک است که با تجهیزات ارزان و کم‌مصرف دیجیتالی پارامترهای جوی لازم را جمع‌آوری و به‌صورت بی‌سیم به ایستگاهی جهت ثبت و نمایش مخابره می‌کند. در این پروژه از میکروکنترلر (\متن‌لاتین{Microcontroller}) \متن‌لاتین{ARM} سری \متن‌لاتین{STM32F10X} به‌عنوان هسته اصلی پردازش در هر دو سمت سنسور و ایستگاه و از ماژول لورا (\متن‌لاتین{LoRa}) با چیپ \متن‌لاتین{SX1278} به‌منظور برقراری ارتباط بی‌سیم استفاده شده ‌است.

\endgroup